\documentclass[12pt,a4paper]{report}
\usepackage[utf8]{inputenc}
\usepackage[T1]{fontenc}
\usepackage[french]{babel}
\usepackage{textcomp}
\usepackage{amsmath,amssymb,amsthm}
\usepackage{lmodern}
\usepackage[a4paper]{geometry}
\usepackage{graphicx}
\usepackage{xcolor}
\usepackage{microtype}
\usepackage{pdfpages}
\usepackage{tikz}
\theoremstyle{plain}

\newtheorem{definition}{Définition}
\newtheorem*{rmq}{Remarque}
\newtheorem{theo}{Théorème}
\newcommand{\enstq}[2]{\left\{#1\,\middle|\,#2\right\}}
\newcommand{\card}[1]{\left\lvert#1\right\rvert}
\newcommand{\ensnombre}[1]{\mathbb{#1}}
\newcommand{\N}[1]{\ensnombre{N}}
\newcommand{\C}[1]{\ensnombre{C}}
\newcommand{\R}[1]{\ensnombre{R}}
\newcommand{\Z}[1]{\ensnombre{Z}}

\usepackage{hyperref}
\hypersetup{pdfstartview=XYZ}

\title{Projet Mandelbrot - Fin de L3}
\author{Camille \textsc{Aimar}, Marc-Antoine \textsc{Mignon-Casseta},\\Njakanavalona \textsc{Rasoloarivony} \\ \\ Université de\emph{ Cergy Pontoise}}
\date{mars 2018}

\begin{document}
\maketitle
\setcounter{chapter}{1}
\section{Introduction}
\begin{itshape}
En mathématiques, \emph{l'ensemble de Mandelbrot} est une fractale définie comme l'ensemble des points $c \in \mathbb{C}$ pour lesquels la suite de nombres complexes définie par récurrence par :
\begin{equation}\label{mandelbrot}
\left\{
\begin{array}{r c l}
z_0 &=& 0\\
z_{n+1} &=& z^2_n+c
\end{array}
\right.
\end{equation}
est bornée.
\setlength{\parskip}{3 mm}

\emph{L'ensemble de Mandelbrot} (notée aussi $\mathcal{M}$) a été découvert par Gaston \textsc{Julia} et Pierre \textsc{Fatou} avant la Première Guerre mondiale. Sa définition et son nom actuel sont dus à Adrien \textsc{Douady}, en hommage aux représentations qu'en a réalisées Benoît \textsc{Mandelbrot} dans les années 1980. Cet ensemble permet d'indicer \emph{les ensembles de Julia} : à chaque point du plan complexe correspond un ensemble de Julia différent. Les points de \emph{l'ensemble de Mandelbrot} correspondent précisément aux \emph{ensembles de Julia} connexes, et ceux en dehors correspondent aux \emph{ensembles de Julia} non connexes. Cet ensemble est donc intimement lié à \emph{l'ensemble de Julia}, ils produisent d'ailleurs des formes similairement complexes.
\setlength{\parskip}{2 mm}

Les images de \emph{l'ensemble de Mandelbrot} sont réalisées en parcourant les nombres complexes sur une région carrée du plan complexe. Les point de cette région du plan complexe se divisent en deux catégories : ceux qui génèrent une suite non bornée et ceux qui génèrent une suite bornée avec la relation de récurrence (\ref{mandelbrot}). On considère la partie réelle et imaginaire de chaque nombre complexe comme des coordonnées. Si le pixel converge, alors il est coloré en noir. Sinon, on le colorie selon sa rapidité de divergence.
\end{itshape}
\newpage
\section{Propriétés}
	\begin{definition}
La théorie générale, développée par Pierre \textsc{Fatou} et Gaston \textsc{Julia} au début du \textsc{XX}\ieme {} siècle associe à toute fonction $f(z,c)$ avec $(z,c) \in \mathbb{C}^2$ les ensembles de Julia $J_C$. Cet ensemble est défini (pour un c fixé) comme la frontière de l'ensemble des $a \in \mathbb{C}$ tels que la suite définie par :
\begin{equation}\label{julia}
\left\{
\begin{array}{r c l}
z_0 &=& a\\
z_{n+1} &=& f(z_n, c)
\end{array}
\right.
\end{equation}
reste bornée en module.
\end{definition}

Pour la fonction particulière $f(z,c)=z^2+c$, on définit l'ensemble de Mandelbrot $\mathcal{M}$ comme l'ensemble des $c \in \mathbb{C}$ pour lequel $J_c$ est connexe.

Cette définition est équivalente à celle donnée dans l'introduction. $c \in \mathcal{M}$ si et seulement si la suite $(z_n)$ définie par (\ref{mandelbrot}) reste bornée (en module).

	\subsection{Critère du module égal à 2}
Il a été démontré que si la suite des modules des $z_n$ est strictement supérieure à $2$ pour un certain indice $N_0$,
($\exists N_0 \in \mathbb{N} \backslash \forall n \in \mathbb{N},{} n>N_0 \Longrightarrow \|z_n|>2$), alors cette suite est croissante à partir de cet indice $N_0$. De plus on a $\lim zn = +\infty$

Une condition simple pour que la suite $(z_n)_{n \in \mathbb{N}}$ soit bornée est de vérifier qu'elle reste bornée par $2$. On vérifie donc que $\forall n \in \mathbb{N}, z_n \subset \mathcal{B}(0,2)$.

	\subsection{Mesures et Dimension de Hausdorff}
	\subsubsection{Mesures extérieures}
On commence par rappeler la définition d'une mesure extérieure sur un sensemble quelconque $E$. Cette notion est un concept, introduit par le mathématicien Constantin \textsc{Carathéodory}. 

\begin{definition}
Soit $X$ un ensemble. Une mesure extérieure sur $X$ est une fonction définie sur l'ensemble de toutes les partie de $X$:
\[\varphi : \mathcal{P}(X) \longrightarrow [0, +\infty\]
qui vérifie les trois conditions suivantes
\end{definition}

\end{document}
